\documentclass[main]{subfiles}
\begin{document}

%@@@@@@@@@@@@@@@@@@@@@@@@@@@@@@
% summarizes lecture 15
% author: Vanessa Leite

\section{Learning Time Series}

\subsection{Stochastic Processes}
Stochastic process is a sequence of random variables
$X_1, . . . , X_n =: (X_t)_t$ with $t \in T$.
Where:
\begin{itemize}
\item $T$ may be finite, countably infinity or continuous
\item The random variables $X_t$ are not assumed to be iid
\item All distribution of th $X_t$ share a common domain $Q$, the state space
\end{itemize}

\subsubsection{Time Series}
Time series (temporal stochastic process) are a sequences where $T$ representing a time axis.
Definitions:
\begin{itemize}
\item Trajectory: random variables sequentially produce a series of numbers
$x_t \in Q$. This sequence is the path/trajectory/realization of the process.
\item Draw sequentially: draw one random variable after another.
\item Draw from a process: drawing a trajectory at once (simultaneously).
\item Consequence: stochastic process as a distribution over trajectories.
\item Powerful model: a trajectory assigns a value $x_t \in Q$ to each $t \in T$.
It is like a function $f: T \to Q$.
\item Famous example: white noise - stochastic process where $X_t$ are iid.
\end{itemize}

\subsubsection{Discrete Stochastic Process}
Restriction: When $T \in N$ and we have a discrete state space.
Stationary Process: a process where
for all $n,k \in N Pr\{X_1 = x_1, . . . , X_n = x_n\} = Pr\{X_1+k = x_1+k, . . . , X_n+k = x_n+k \}$
Meaning: the joint distribution is invariant to index shifts.

\subsection{Markov Models}
\subsubsection{Markov Process}
The outcome of each draw depends only on the previous one (memoryless or Markov Chain).
$Pr\{X_t = x_t| x_1, . . . , x_n\} = Pr\{X_t = x_t | x_t-1\}$

\subsubsection{Stationary Markov Chains}
In the stationary sequential distribution the data evolves in time, but the distribution from
which it is generated remains the same.

Finite state space: Given $Q = \{x_1, . . . , x_n\}$. Then we define:
$a_ij(t) = Pr\{X_t = x_j | X_t-1 = x_i\}$
So, $A(t)$ is a $n x n$ matrix for each $t$.
Meaning: $a_ij$ is the probability to change from state $i$ to state $j$
Given the matrix multiplication: $A^2 = A * A$, so, $a_{ij}^2 = \sum a_{il} * a_{lk} $ with $1 \leq l \leq n$.
Meaning: $a_{ij}^2$ is the probability to change from state $i$ to state $j$ in two steps. In general:
$a_{ij}^k$ is the probability to change from state $i$ to state $j$ in $k$ steps.

\subsubsection{Graph representation}
$A$ is a $n x n$ matrix with non-negative entries.
$A = G = (V,E)$ where
\begin{itemize}
\item Nodes are states
\item Edges weights are transition probabilities
\end{itemize}
If we include a start state and an end state, we have a finite automaton (finite state machine).
A finite stationary markov chain is a triplet $(Q, p, A)$, where:
\begin{itemize}
\item $Q$ is a finite set of states
\item $p$ is a vector of initial probabilities (probability of each state be the initial state)
\item $A$ is a transition matrix
\end{itemize}
\subsection{Hidden Markov Models}
While in the Markov Chain each state has a symbol (and reach a state generates that symbol),
in the HMM each state can generate different symbols, each one with certain probability.
Formal definition: HMM is a triplet $M = (\sigma, Q, \theta)$, where:
\begin{itemize}
\item $\sigma$ is the alphabet of symbols
\item $Q$ is a finite set of states
\item $\theta$ is a set of probabilities, with a transition matrix $A$
 and a matrix $E$ of emission probability $e_k (b)$ for each $x_k \in Q$ and $b \in \sigma$.
\end{itemize}

Observed symbols: for each index $t \in T$ we introduce random variables $S_t$,
with domain $\sigma$ that encode the observed symbol.

Emission probabilities: probability that the symbol $s_t$ is seen when we are in state $x_k$:
$e_k (s_t) = P(S_t = s_t | X_t = x_k)$

Combining probabilities: Let $x = (x_1, . . . , x_n)$ be a path and $s = (s_1, . . . , s_n)$ be a string of symbols.
The probability of drawing a path $x$ with $x_0 = B$ and $x_n+1 = \varepsilon$ and generating the string s is:
$P(s, x) = a_{x0x1} * \prod e_{xt}(s_t) * a_{xtxt+1}$

\subsubsection{Three basic problems for HMM}
\begin{itemize}
\item Evaluation: Find the probability that a given sequence $s$ was emitted by the HMM.
\item Decoding: Find the most likely path $x$ responsible for the given sequence $s$.
\item Learning: Find the parameters of HMM ($a_{ij}, e_{k}(s_{t})$ and the path $x$)
\end{itemize}

\subsection{Classifying Animal Behavior with HMM}
\end{document}
